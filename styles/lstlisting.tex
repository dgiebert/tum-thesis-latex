% Using Light Monokai theme

% See: https://tex.stackexchange.com/questions/47175/scala-support-in-listings-package
\lstdefinelanguage{scala}{
    morekeywords={abstract,case,catch,class,def,%
      do,else,extends,false,final,finally,%
      for,if,implicit,import,match,mixin,%
      new,null,object,override,package,%
      private,protected,requires,return,sealed,%
      super,this,throw,trait,true,try,%
      type,val,var,while,with,yield},
    otherkeywords={=>,<-,<\%,<:,>:,\#,@},
    sensitive=true,
    morecomment=[l]{//},
    morecomment=[n]{/*}{*/},
    morestring=[b]",
    morestring=[b]',
    morestring=[b]"""
  }

\lstdefinestyle{scala}{
    xleftmargin=.05\textwidth,               % Center the listing
    xrightmargin=.05\textwidth,              % Center the listing
    frame=lrb,                        % Draw the shadowbox as frame
    belowcaptionskip=-1pt,
    rulecolor=\color{TUMBlue},                % Color of the frame
    rulesepcolor=\color{TUMGray!50},        % Color of the shadow
    language=scala,                			% choose the language of the code
    numbers=left,                   		% where to put the line-numbers
    stepnumber=1,                   		% the step between two line-numbers.        
    numbersep=5pt,                  		% how far the line-numbers are from the code
    numberstyle=\tiny\color{TUMGray},       % gray and tiny numbers
    backgroundcolor=\color{background},  	% choose the background color. You must add \usepackage{color}
    showspaces=false,               		% show spaces adding particular underscores
    showstringspaces=false,         		% underline spaces within strings
    showtabs=false,                 		% show tabs within strings adding particular underscores
    tabsize=4,                      		% sets default tabsize to 4 spaces
    captionpos=t,                   		% sets the caption-position to bottom
    breaklines=true,                		% sets automatic line breaking
    breakatwhitespace=true,         		% sets if automatic breaks should only happen at whitespace
    title=\lstname,                 		% show the filename of files included with \lstinputlisting;
    basicstyle=\color{normal}\ttfamily,		% sets font style for the code
    keywordstyle=\color{keyword}\ttfamily,	% sets color for keywords
    stringstyle=\color{string}\ttfamily,	% sets color for strings
    commentstyle=\color{comment}\ttfamily,	% sets color for comments
    procnamekeys={def, object, class},      % Highlight the function, class, object name
    procnamestyle=\color{function}          % Color of the highlight
}

\DeclareCaptionFont{white}{\color{white}} 
\DeclareCaptionFormat{listing}{% 
  \colorbox{TUMBlue}{\parbox{\dimexpr0.9\textwidth+2\fboxsep-5pt\relax}{#1#2#3}}} 
\captionsetup[lstlisting]{format=listing,labelfont=white,textfont=white} 