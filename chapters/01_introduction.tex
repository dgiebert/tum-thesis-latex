% !TeX root = ../main.tex
% Add the above to each chapter to make compiling the PDF easier in some editors.

\chapter{Introduction}\label{chapter:introduction}

\section{Section}
Citation test~\parencite{latex}.

\subsection{Subsection}


\begin{wraptable}{r}{5.5cm}
    \centering
    \begin{tabular}{l l l l}
      \toprule
      \multicolumn{1}{c}{\textbf{A}\footnotemark} & \textbf{B} & \textbf{C} & \textbf{D} \\
      \midrule
        1\footnotemark & 2 & 1 & 2 \\ \rowcolor{TUMAccentLightBlue!30}
        2 & 3 & 2 & 3 \\
        2 & 3 & 2 & 3 \\
      \bottomrule
    \end{tabular}
    \caption[Example table]{An example for a wrapped table.}\label{tab:sample}
  \end{wraptable}
  \addtocounter{footnote}{-1}
  \footnotetext{\url{www.google.de}}
  \addtocounter{footnote}{1}
  \footnotetext{\url{www.tum.de}}
See~\autoref{tab:sample}, \autoref{fig:sample-drawing}, \autoref{fig:sample-plot}, \autoref{fig:sample-listing}.

\lipsum[1-2]


\begin{figure}[htpb]
  \centering
  % This should probably go into a file in figures/
  \begin{tikzpicture}[node distance=3cm]
    \node (R0) {$R_1$};
    \node (R1) [right of=R0] {$R_2$};
    \node (R2) [below of=R1] {$R_4$};
    \node (R3) [below of=R0] {$R_3$};
    \node (R4) [right of=R1] {$R_5$};

    \path[every node]
      (R0) edge (R1)
      (R0) edge (R3)
      (R3) edge (R2)
      (R2) edge (R1)
      (R1) edge (R4);
  \end{tikzpicture}
  \caption[Example drawing]{An example for a simple drawing.}\label{fig:sample-drawing}
\end{figure}

\begin{figure}[htpb]
  \centering

  \pgfplotstableset{col sep=&, row sep=\\}
  % This should probably go into a file in data/
  \pgfplotstableread{
    a & b    \\
    1 & 1000 \\
    2 & 1500 \\
    3 & 1600 \\
  }\exampleA
  \pgfplotstableread{
    a & b    \\
    1 & 1200 \\
    2 & 800 \\
    3 & 1400 \\
  }\exampleB
  % This should probably go into a file in figures/
  \begin{tikzpicture}
    \begin{axis}[
        ymin=0,
        legend style={legend pos=south east},
        grid,
        thick,
        ylabel=Y,
        xlabel=X
      ]
      \addplot table[x=a, y=b]{\exampleA};
      \addlegendentry{Example A};
      \addplot table[x=a, y=b]{\exampleB};
      \addlegendentry{Example B};
    \end{axis}
  \end{tikzpicture}
  \caption[Example plot]{An example for a simple plot.}\label{fig:sample-plot}
\end{figure}

\begin{figure}[htpb]
  \centering
  \begin{tabular}{c}
  \begin{lstlisting}[language=SQL]
    SELECT * FROM tbl WHERE tbl.str = "str"
  \end{lstlisting}
  \end{tabular}
  \caption[Example listing]{An example for a source code listing.}\label{fig:sample-listing}
\end{figure}

\begin{minipage}[t]{\textwidth}
\begin{lstlisting}[style=scala, caption=Simple Listing example]
object Test {
    def main(args: Array[String]) {
        var a = 0;
        // for loop execution with a range
        for( a <- 1 to 10){
            println( "Value of a: " + a );
        }
    }
}
\end{lstlisting}
\label{overview}
\end{minipage}